\usepackage[utf8]{inputenc}
\usepackage[T1]{fontenc}
\usepackage{lipsum}% juste utile ici pour générer du faux texte}
\usepackage{mwe}%juste utile ici pour générer de fausses images
\usepackage{amsmath,amsfonts,amssymb}%extensions de l'ams pour les mathé matiques
\usepackage{shorttoc}%pour la réalisation d'un sommaire.
\usepackage{tikz}
\usepackage{graphicx}%pour insérer images et pdf entre autres
	\graphicspath{{images/}}%pour spécifier le chemin d'accès aux images
\usepackage[left=3.5cm,right=2.5cm,top=4cm,bottom=4cm]{geometry}%réglages des marges du document selon vos préférences ou celles de votre établissemant
\usepackage[Lenny]{fncychap}%pour de jolis titres de chapitres voir la doc pour d'autres styles.

\usepackage{fancyhdr}%pour les entêtes et pieds de pages
	\setlength{\headheight}{14.2pt}% hauteur de l'entête

%%%%%%%%%%%%%%%%%%%style front%%%%%%%%%%%%%%%%%%%%%%%%%%%%%%%%%%%%%%%%%	
	\fancypagestyle{front}{%
  		\fancyhf{}%on vide les entêtes
  		\fancyfoot[C]{page \thepage}%
  		\renewcommand{\headrulewidth}{0pt}%trait horizontal pour l'entête
  		\renewcommand{\footrulewidth}{0.4pt}%trait horizontal pour les pieds de pages
		}


%%%%%%%%%%%%%%%%%%%style main%%%%%%%%%%%%%%%%%%%%%%%%%%%%%%%%%%%%
	\fancypagestyle{main}{%
		\fancyhf{}
  		\renewcommand{\chaptermark}[1]{\markboth{\chaptername\ \thechapter.\ ##1}{}}% redéfintion pour avoir ici les titres des chapitres des sections en minuscules
  		\renewcommand{\sectionmark}[1]{\markright{\thesection\ ##1}}
		\fancyhead[c]{}
		\fancyhead[RO,LE]{\rightmark}%
  		\fancyhead[LO,RE]{\leftmark}
		\fancyfoot[C]{}
		\fancyfoot[RO,LE]{page \thepage}%
  		\fancyfoot[LO,RE]{Mon rapport}
  		}

%%%%%%%%%%%%%%%%%%%style back%%%%%%%%%%%%%%%%%%%%%%%%%%%%%%%%%%%%%%%%%	
	\fancypagestyle{back}{%
  		\fancyhf{}%on vide les entêtes
  		\fancyfoot[C]{page \thepage}%
  		\renewcommand{\headrulewidth}{0pt}%trait horizontal pour l'entête
  		\renewcommand{\footrulewidth}{0.4pt}%trait horizontal pour les pieds de pages
		}


%%%%%%%%%%%%%%%%%%%%%%%%%%%%index%%%%%%%%%%%%%%%%%%%%%%%%%%%%%%%%%%%%%%%
\usepackage{makeidx}
\makeindex


%%%%%%%%%%%%%%%%%%%%%%%%%%%%%glossaire%%%%%%%%%%%%%%%%%%%%%%%%%%%%%%%%%%%
\usepackage{glossaries}
%\makeglossaries

\usepackage[english,french]{babel}%pour un document en français
\usepackage{hyperref}%rend actif les liens, références croisée, toc, ...
		\hypersetup{colorlinks,%
		citecolor=black,%
		filecolor=black,%
		linkcolor=black,%
		urlcolor=black} 


%%%%%%%%%%%%%%%%%%%%%%%%%%%%biblio%%%%%%%%%%%%%%%%%%%%%%%%%%%%%%%%%%%%%%
\usepackage[backend=biber]{biblatex}
\addbibresource{bibliographie/biblio.bib}% pour indiquer ou se trouve notre .bib
\usepackage{csquotes}% pour la gestion des guillemets français.

%%%%%%%%%%%%%%%%%%%%%%%%%%%%%glossaire%%%%%%%%%%%%%%%%%%%%%%%%%%%%%%%%%%%
\usepackage{glossaries}
\makeglossaries		

%%%%%%%%%%%%%%%%%%%%%%%%%%%%liste des abbréviations%%%%%%%%%%%%%%		
\usepackage[french]{nomencl}
\makenomenclature
\renewcommand{\nomname}{Liste des abréviation, des sigles et des symboles}



\makeatletter
\newenvironment{abstract}{%
    \cleardoublepage
    \null\vfil
    \@beginparpenalty\@lowpenalty
    \begin{center}%
      \bfseries \abstractname
      \@endparpenalty\@M
    \end{center}}%
   {\par\vfil\null}
\makeatother
